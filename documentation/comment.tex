\begin{comment}
\\
\nnu_{\nt}=& \frac{1}{2\Delta \nt} (\nnu{_{\nni}^{\nm+1}} - \nnu{_{\nni}^{\nm-1}} ) \nonumber\\
\nnu_{\nx} = &\frac{1}{2\Delta \nx} (\nnu{_{\nni+1}^{\nm}} - \nnu{_{\nni-1}^{\nm}} ) \nonumber\\
\nnu _{\nx \nx \nx } =& \frac{1}{2\Delta \nx^3} (\nnu{_{\nni+2}^{\nm}} - 2\nnu{_{\nni+1}^{\nm}} + 2\nnu{_{\nni-1}^{\nm}}- \nnu{_{\nni-2}^{\nm}} ) \nonumber \\
\nnu _{\nt} + 6\nnu \nnu _{\nx} + \nnu _{\nx \nx \nx} = & \frac{1}{2\Delta \nt}  (\nnu{_{\nni}^{\nm+1}} - \nnu{_{\nni}^{\nm-1}} ) + \frac{1}{\Delta \nx} (\nnu{_{\nni-1} ^{\nm}} + \nnu{_{\nni}^{\nm}} + \nnu{_{\nni+1}^{\nm}} ) (\nnu{_{\nni+1}^{\nm}} - \nnu{_{\nni-1}^{\nm}} ) +  \nonumber \\
&       + \frac{1}{2\Delta \nx^3} (\nnu{_{\nni+2}^{\nm}} - 2\nnu{_{\nni+1}^{\nm}} + 2\nnu{_{\nni-1}^{\nm}}- \nnu{_{\nni-2}^{\nm}} ) = 0 \label{eq:NumZabKru}

Program

fouriertrafo

Fourier-Reihen, Teil 6b – DFT gemessener Signale – Herr Fessa
https://herrfessa.com/2018/11/25/fourier-reihen-teil-6b-dft-gemessener-signale/
Fourier-Reihen, Teil 3 – Die Berechnung des Spektrums – Herr Fessa
https://herrfessa.com/2018/03/06/fourier-reihen-teil-3-die-berechnung-des-spektrums/

Diskrete Fourier-Transformation – Wikipedia
https://de.wikipedia.org/wiki/Diskrete_Fourier-Transformation
Fourierreihe – Wikipedia
https://de.wikipedia.org/wiki/Fourierreihe
Discrete Fourier Transform - Simple Step by Step - YouTube
https://www.youtube.com/watch?v=mkGsMWi_j4Q


random values in c++
-----
set the following command before the random function
-srand(time(NULL));

stold - C++ Reference
http://www.cplusplus.com/reference/string/stold/
How to parse command line parameters. - C++ Articles
http://www.cplusplus.com/articles/DEN36Up4/
Command line arguments in C/C++ - GeeksforGeeks
https://www.geeksforgeeks.org/command-line-arguments-in-c-cpp/
7.13 — Command line arguments | Learn C++
https://www.learncpp.com/cpp-tutorial/713-command-line-arguments/
How to parse command line parameters. - C++ Articles
http://www.cplusplus.com/articles/DEN36Up4/
How to Parse Command Line Arguments in C++? - Stack Overflow
https://stackoverflow.com/questions/865668/how-to-parse-command-line-arguments-in-c
Chapter 31. Boost.Program_options - 1.69.0
https://www.boost.org/doc/libs/1_69_0/doc/html/program_options.html
<random> - C++ Reference
http://www.cplusplus.com/reference/random/
default_random_engine - C++ Reference
http://www.cplusplus.com/reference/random/default_random_engine/
normal_distribution - C++ Reference
http://www.cplusplus.com/reference/random/normal_distribution/
Getting Started · Distributions.jl
https://juliastats.github.io/Distributions.jl/stable/starting.html
10. API Reference · Pkg.jl
https://julialang.github.io/Pkg.jl/v1/api/#API-Reference-1
seeding c++ random values - Ecosia
https://www.ecosia.org/search?q=seeding+c%2B%2B+random+values&ref=waterfox
c++ - What is a seed in terms of generating a random number? - Stack Overflow
https://stackoverflow.com/questions/14914595/what-is-a-seed-in-terms-of-generating-a-random-number
srand - C++ Reference
http://www.cplusplus.com/reference/cstdlib/srand/
Random number generation seeding in C++ - Code Review Stack Exchange
https://codereview.stackexchange.com/questions/101525/random-number-generation-seeding-in-c
class - Error "C++ requires a type specifier for all declarations whilst defining methods" - Stack Overflow
https://stackoverflow.com/questions/19800091/error-c-requires-a-type-specifier-for-all-declarations-whilst-defining-method



----

<random> - C++ Reference
http://www.cplusplus.com/reference/random/?kw=random
default_random_engine - C++ Reference
http://www.cplusplus.com/reference/random/default_random_engine/
normal_distribution - C++ Reference
http://www.cplusplus.com/reference/random/normal_distribution/
GitHub - vgvassilev/cling: The interactive C++ interpreter Cling
https://github.com/vgvassilev/cling
Stochastic dynamical systems - Scholarpedia
http://www.scholarpedia.org/article/Stochastic_dynamical_systems
Synchronization - Scholarpedia
http://www.scholarpedia.org/article/Synchronization
Phase Model - Scholarpedia
http://www.scholarpedia.org/article/Phase_models
Search results for "autocorrelation" - Scholarpedia
http://www.scholarpedia.org/w/index.php?search=autocorrelation&title=Special%3ASearch
Van der Pol oscillator - Scholarpedia
http://www.scholarpedia.org/article/Van_der_Pol_oscillator
Dynamical systems - Scholarpedia
http://www.scholarpedia.org/article/Dynamical_systems
Bifurcation - Scholarpedia
http://www.scholarpedia.org/article/Bifurcation
Fluctuations - Scholarpedia
http://www.scholarpedia.org/article/Fluctuations
Autokorrelation – Wikipedia
https://de.wikipedia.org/wiki/Autokorrelation
Durbin-Watson-Test – Wikipedia
https://de.wikipedia.org/wiki/Durbin-Watson-Test
Correlogram - Wikipedia
https://en.wikipedia.org/wiki/Correlogram
Correlation and dependence - Wikipedia
https://en.wikipedia.org/wiki/Correlation_and_dependence
Convolution - Wikipedia
https://en.wikipedia.org/wiki/Convolution
Convolution (computer science) - Wikipedia
https://en.wikipedia.org/wiki/Convolution_(computer_science)
Cross-correlation - Wikipedia
https://en.wikipedia.org/wiki/Cross-correlation
Autocorrelation
https://www.investopedia.com/terms/a/autocorrelation.asp
Autocorrelation - Statistics Solutions
https://www.statisticssolutions.com/autocorrelation/
* Autokorrelation | Statista
https://de.statista.com/statistik/lexikon/definition/28/autokorrelation/
A Gentle Introduction to Autocorrelation and Partial Autocorrelation
https://machinelearningmastery.com/gentle-introduction-autocorrelation-partial-autocorrelation/
Einführung in die Zeitreihenanalyse | SpringerLink
https://link.springer.com/book/10.1007/3-540-33571-4
Gaussian noise - Wikipedia
https://en.wikipedia.org/wiki/Gaussian_noise
Uncorrelatedness (probability theory) - Wikipedia
https://en.wikipedia.org/wiki/Uncorrelatedness_(probability_theory)
Correlation and dependence - Wikipedia
https://en.wikipedia.org/wiki/Correlation_and_dependence
Additives weißes gaußsches Rauschen – Wikipedia
https://de.wikipedia.org/wiki/Additives_wei%C3%9Fes_gau%C3%9Fsches_Rauschen
Euler-Maruyama-Verfahren – Wikipedia
https://de.wikipedia.org/wiki/Euler-Maruyama-Verfahren
Stochastische Differentialgleichung – Wikipedia
https://de.wikipedia.org/wiki/Stochastische_Differentialgleichung
Wiener-Prozess – Wikipedia
https://de.wikipedia.org/wiki/Wiener-Prozess
Stochastic Differential Equations | SpringerLink
https://link.springer.com/book/10.1007/978-3-642-14394-6
Spectral density - Wikipedia
https://en.wikipedia.org/wiki/Spectral_density
Euler-Maruyama-Verfahren – Wikipedia
https://de.wikipedia.org/wiki/Euler-Maruyama-Verfahren


\nwfilename{stoch_reso.nw}\nwbegindocs{1}\nwdocspar
@language c++
void documentation()
{
  // Protocol
  // starting point
  // init(T, A, omega, sigma, N,switchi);
  //init(150, 1, 0.09, 0.5, 1E5,1);
  // working parameters for kramers rule
  //init(200, 0, 0.2, 0.4, 1E5,0);
  // testing kramers rule
  //init(500, 0.5, 0.2, 0.4, 1E5,1);
  // task 4 calculating the amplitude of the fouriercomponents for every sigma
  // investigate 
  init(500, 0.2, 100, 1.0, 1E6,1);
  //x.pushback(stold(argv[6])); // initial value
  x.push_back(0); //startvalue =0
}
@language python
from subprocess import call
# Methode zum Erzeugen des Programm Quellcodes
g.execute_shell_commands("echo '----------------Start Programm----------------'")
#g.execute_shell_commands("notangle -R***.*** ***.nw > ***.*** &")

##### Meta
'''
programming language: C++
methods:
        - gaussian white noise
        - stochastic method solver Euler-Maruyama-method
'''

##### History
'''
7.2.2019
        - Changing of the program code for calculating the Euler-Maruyama-method
                - changing from deltaW = W_i-W-i-1 to deltaW = gausnoise(0,1)
                - two new outputs and plotting 
                        - output T_Omega/2=T_K
                        - output T_K=1/r_K


3.2.2019
        - start of writing the protocol
                - started sections: 
                        - general equations 
                        - numerical method
        - start of implementing Kramers formula
        - checking of the nonlinear properties of the diff equation 
                - calculating the mean value of x: bar(x)

29.1.2019
        - rewrite of the parameter delta_t = T/N
                - delta_t is now calculated through delta_t = T/N
        - new parameter N, before tstep now N
        - T is the first parameter , duration of the simulation
        - N - number if calculations
        - the white noise calculated with mu = 0 and variance = sqrt(tstep)

23.1.2019
        - insert random function generateGaussianNoise
        - fix of the random values
        - finding of a state or initial values that brings the time series changing only with the noise, creating of a new oscillation 

21.01.2019
        - Überarbeitung des Programms
                - Variable T entfernt
                - Einfügen von Shellparameter
                - das Lösungschemata der Differentialgleichung richtig implementiert
                - steps ist jetzt tstep
                - while zu for schleife umgewandelt
                - number variable ist ein array --> dadurch extrem schneller geworden
                - periods ist nun omega also jetzt so wie in der Formel formuliert und eingesetzt
                        - periods bleibt für eventuell weitere Sachen noch erhalten
                
                
18.01.2019
        - Start Programmierung
        - Überarbeitung des Programms von Alex
    
'''


##### Doku
## Parameter
'''
p1 - T - duration of simulation
p2 - A - A, amplitude of the oscillator
p3 - omega - omega, oscillation of the oscillatior
p4 - sigma - sigma, amplitude of the noise
p5 - N - number of calculations
p6 - x_0, start value of x

'''
## Programm ausführen

## Schwingung mit diesen Parametern here with 1/tstep
# 40 1 1 0.05 10000 0
## stochastische Werte bringen die Zeitreihe mal zum kippen, es entsteht eine neue Schwingung, die Zufallswerte übersteigen manchmal den einen Bereich und es kippt in die andere Seite um.
# 1000 0.35 0.1 0.2 10000 1

#g.execute_shell_commands("clang++ Main.cpp -o stoch_reso && time ./stoch_reso 100 -1.8 0.8 0.4 10000 0.6")#
#g.execute_shell_commands("clang++ Main.cpp -o stoch_reso && time ./stoch_reso 1000 0 0.3 0.3 10000 0")#

###### here calcluating with T, N, tstep = T/N

## task 0: Testing of A = 0, so we get a random switch between the states 
#g.execute_shell_commands("clang++ Main.cpp -o stoch_reso && time ./stoch_reso 500 0 0.3 0.5 100000 0.1")

## Probing A >0
#g.execute_shell_commands("clang++ Main.cpp -o stoch_reso && time ./stoch_reso 150 0 0.09 0.6 1000000 0")

## task 4 nonlinear properties of the diff equation
# at the amplitude A = 0.40102 it switches randomly between the two states
# maybe programming a for-loop for the A amplitude depending on the mean of x, diagram of A-bar(x) 
#g.execute_shell_commands("clang++ Main.cpp -o stoch_reso && time ./stoch_reso 140 0.40102 0.05 6 10000000 0")

##Alexs program testing
g.execute_shell_commands("clang++ Main_Alex_synth.cpp -o stoch_reso_alex_synth && time ./stoch_reso_alex_synth")
#g.execute_shell_commands("clang++ Main.cpp -o stoch_reso && time ./stoch_reso")
#g.execute_shell_commands("clang++ Main_Alex_newest.cpp -o stoch_reso_alex_newest && time ./stoch_reso_alex_newest")

### Plot der Daten

g.execute_shell_commands("echo '----------------Gnuplot Start----------------'")
g.execute_shell_commands("gnuplot plotting.plt -p &")
g.execute_shell_commands("gnuplot plotting-a_k-b_k.plt -p &")
g.execute_shell_commands("gnuplot plotting-ampl.plt -p &")
g.execute_shell_commands("echo '----------------Gnuplot Ende ----------------'")


\nwenddocs{}\nwbegindocs{2}\nwdocspar
@language python
##### Programm zum Start der Versionisierung des Codes
from subprocess import call
g.execute_shell_commands("notangle -RVersions.conf stoch_reso.nw > Versions.conf &")
g.execute_shell_commands("/home/christian/Programme/storeBackup/bin/storeBackup.pl -f Versions.conf &")

\nwenddocs{}\nwbegindocs{3}\nwdocspar

@language maple
\nwenddocs{}\nwbegincode{4}\sublabel{NW4PIMRJ-3YPuMQ-1}\nwmargintag{{\nwtagstyle{}\subpageref{NW4PIMRJ-3YPuMQ-1}}}\moddef{Versions.conf~{\nwtagstyle{}\subpageref{NW4PIMRJ-3YPuMQ-1}}}\endmoddef\nwstartdeflinemarkup\nwenddeflinemarkup

# configuration file for storeBackup.pl
# Generated by storeBackup.pl, 3.5

####################
### explanations ###
####################

# You can set a value specified with '-cf_key' (eg. logFiles) and
# continue at the next lines which have to begin with a white space:
# logFiles = /var/log/messages  /var/log/cups/access_log
#      /var/log/cups/error_log
# One ore more white spaces are interpreted as separators.
# You can use single quotes or double quotes to group strings
# together, eg. if you have a filename with a blank in its name:
# logFiles = '/var/log/my strage log'
# will result in one filename, not in three.
# If an option should have *no value*, write:
# logFiles =
# If you want the default value, comment it:
logFile = Versions-Programm.log

# You can also use environment variables, like $XXX or $\{XXX\} like in
# a shell. Single quotes will mask environment variables, while double
# quotes will not.
# You can mask $, \{, \}, ", ' with a backslash (\\), eg. \\$
# Lines beginning with a '#' or ';' are ignored (use this for comments)
#
# You can overwrite settings in the command line. You can remove
# the setting also in the command by using the --unset feature, eg.:
# '--unset doNotDelete' or '--unset --doNotDelete'

#####################
### configuration ###
#####################

# source directory (*** must be specified ***)
sourceDir=/home/christian/Dropbox/Comp-Physics-3/

# top level directory of all linked backups (*** must be specified ***)
# storeBackup must know for consistency checking where all your backups
# are. This is done to make sure that your backups are consistent if you
# used --lateLinks.
backupDir=/home/christian/Dropbox/Comp-Physics-3/

# ------------------------------------------------------------------------
# you do not need specify the options below to get a running configuration
# (but they give you more features and more control)
#


# series directory, default is 'default'
# relative path from backupDir
series=Versions

# directory for temporary file, default is /tmp
;tmpDir=

# List of other backup directories to consider for
# hard linking. Relative path from backupDir!
# Format (examples):
# backupSeries/2002.08.29_08.25.28 -> consider this backup
# or
# 0:backupSeries    -> last (youngest) backup in <backupDir>/backupSeries
# 1:backupSeries    -> first before last backup in <backupDir>/backupSeries
# n:backupSeries    -> n'th before last backup in <backupDir>/backupSeries
# 3-5:backupSeries  -> 3rd, 4th and 5th in <backupDir>/backupSeries
# all:backupSeries  -> all in <backupDir>/backupSeries
# This option is useful, if you want to explicitly hard link
# to backup series from different backups. You can specify eg. with
# 0:myBackup to the last backup of series 'myBackup'. If you specify
# backup series with otherBackupSeries, then only these backups will be
# used for hard linking.
# You can also use wildcards in series names. See documentation,
# section 'Using Wildcards for Replication' for details.
# Default value is to link to the last backup of all series stored in
# 'backupDir'.
;otherBackupSeries=

# lock file, if exist, new instances will finish if
# an old is already running, default is /tmp/storeBackup.lock
;lockFile=

# remove the lock files before deleting old backups
# default ('no') is to delete the lock file after deleting
# possible values are 'yes' and 'no'
;unlockBeforeDel=

# continue if one or more of the exceptional directories
# do not exist (no is stopping processing)
# default is 'no', can be 'yes' or 'no'
;contExceptDirsErr=

# Directories to exclude from the backup (relative path inside of the backup).
# You can use shell type wildcards.
# These directories have to be separated by space or newline.
exceptDirs=Versions

# Directories to include in the backup (relative path inside of the backup).
# You can use shell type wildcards.
# These directories have to be separated by space or newline.
;includeDirs=

# rule for excluding files / only for experienced administrators
# !!! see README file 'including / excluding files and directories'
# EXAMPLE: 
# searchRule = ( '$size > &::SIZE("3M")' and '$uid eq "hjc"' ) or
#    ( '$mtime > &::DATE("3d4h")' and not '$file =~ m#/tmp/#' )'
;exceptRule=

# For explanations, see 'exceptRule'.
;includeRule=

# write a file name .storeBackup.notSaved.bz2 with the
# names of all skipped files, default is 'no', can be 'yes' or 'no'
;writeExcludeLog=

# do not save the specified types of files, allowed: Sbcfpl
# S - file is a socket
# b - file is a block special file
# c - file is a character special file
# f - file is a plain file
# p - file is a named pipe
# l - file is a symbolic link
# Spbc can only be backed up if GNU copy is available.
;exceptTypes=

# save the specified type of files in an archive instead saving
# them directly in the file system
# use this if you want to backup those file types but your target
# file or transport (eg. sshfs or non gnu-cp) system does not support
# those types of file
#   S - file is a socket
#   b - file is a block special file
#   c - file is a character special file
#   p - file is a named pipe
#   l - file is a symbolic link
# you also have to set specialTypeArchiver when using this option
;archiveTypes=


# possible values are 'cpio', 'tar', 'none'. default is 'cpio'
# tar is not able to archive sockets
# cpio is not part of the actual posix standard any more
;specialTypeArchiver=

# Activate this option if your system's cp is a full-featured GNU
# version. In this case you will be able to also backup several
# special file types like sockets.
# Possible values are 'yes' and 'no'. Default is 'no'
cpIsGnu=yes

# make a hard link to existing, identical symlinks in old backups
# use this, if your operating system supports this (linux does)
# Possible values are 'yes' and 'no'. Default is 'no'
;linkSymlinks=

# exec job before starting the backup, checks lockFile (-L) before
# starting (e.g. can be used for rsync) stops execution if job returns
# exit status != 0
;precommand=

# exec job after finishing the backup, but before erasing of old
# backups reports if job returns exit status != 0
;postcommand=

# follow symbolic links like directories up to depth 0 -> do not
# follow links
;followLinks=

# only store the contents of file systems named by
# sourceDir and symlinked via followLinks
# possible values are 'yes' and 'no'; default is 'no'
;stayInFileSystem=

# use this only if you write your backup over a high latency line
# like a vpn over the internet
# storebackup will use more parallelization at the cost of more
# cpu power
# possible values are 'yes' and 'no'; default is 'no'
;highLatency=

# If this option is disabled, then the files in the backup will not
# neccessarily have the same permissions and owner as the originals.
# This speeds up backups on network drives a lot. Correct permissions
# are restored by storeBackupRecover.pl no matter what this option is
# set to. Default is 'no'
;ignorePerms=

# suppress (unwanted) warnings in the log files;
# to suppress warnings, the following keys can be used:
#   excDir (suppresses the warning that excluded directories
#          do not exist)
#   fileChange (suppresses the warning that a file has changed during
#              the backup)
#   crSeries (suppresses the warning that storeBackup had to create the
#            'default' series)
#   hashCollision (suppresses the warning if a possible
#                 hash collision is detected)
#   fileNameWithLineFeed (suppresses the warning if a filename
#                        contains a line feed)
#    use_DB_File (suppresses the warning that you should install
#                 perl module DB_File for better perforamnce)
#    use_MLDBM (suppresses the warning that you should install
#               perl module MLDBM if you want to use rule functions
#               MARK_DIR or MARK_DIR_REC together with option saveRAM)
#    use_IOCompressBzip2 (suppresses the warning that you should
#                         instal perl module IO::Compress::Bzip2
#                         for better performance)
#    noBackupForPeriod (suppresses warning that there are
#                       no backups for certain periods when using
#                       option keepRelative)
#  This option can be repeated multiple times on the command line.
#  Example usage in conf file:
#  suppressWarning = excDir fileChange crSeries hashCollision
#  By default no warnings are suppressed.
;suppressWarning=

# do *not* write hard links to existing files in the backup
# during the backup (yes|no)
# you have to call the program storeBackupUpdateBackup.pl
# later on your server if you set this flag to 'yes'
# you have to run storeBackupUpdateBackup.pl later - see
# description for that program
# default = no: do not write hard links
;lateLinks=

# only in combination with --lateLinks
# compression from files >= size will be done later,
# the file is (temporarily) copied into the backup
# default = no: no late compression
;lateCompress=

# repair simple inconsistencies (from lateLinks) automatically
# without requesting the action
# default = no, no automatic repair
;autorepair=

# Files with specified suffix for which storeBackup will make an md5 check
# on blocks of that file. Executed after --checkBlocksRule(n)
;checkBlocksSuffix=

# Only check files specified in --checkBlocksSuffix if there
# file size is at least this value, default is 100M
;checkBlocksMinSize=

# Block size for files specified with --checkBlocksSuffix
# default is 1M (1 megabyte)
;checkBlocksBS=

# if set, the blocks generated due to checkBlocksSuffix are compressed
# Possible values are 'check, 'yes' and 'no'. Default is 'no'
# check uses COMRESSION_CHECK (see option compressSuffix)
;checkBlocksCompr=

# Read files specified here in parallel to "normal" ones.
# This only makes sense if they are on a different disk.
# Default value is 'no'
;checkBlocksParallel=

# length of queue to store files before block checking,
# default = 1000
;queueBlock=

# Files for which storeBackup will make an md5 check depending
# on blocks of that file.
# The rules are checked from rule 1 to rule 5. The first match is used
# !!! see README file 'including / excluding files and directories'
# EXAMPLE: 
# searchRule = ( '$size > &::SIZE("3M")' and '$uid eq "hjc"' ) or
#    ( '$mtime > &::DATE("3d4h")' and not '$file =~ m#/tmp/#' )'
;checkBlocksRule0=

# Block size for option checkBlocksRule
# default is 1M (1 megabyte)
;checkBlocksBS0=

# if set to 'yes', blocks generated due to this rule will be compressed
# possible values: 'check', 'yes' or 'no', default is 'no'
# check users COMRESSION_CHECK (see option compressSuffix)
;checkBlocksCompr0=

# Filter for reading the file to treat as a blocked file
# eg.   gzip -d   if the file is compressed. Default is no read filter.
;checkBlocksRead0=

# Read files specified here in parallel to "normal" ones.
# This only makes sense if they are on a different disk.
# Default value is 'no'
;checkBlocksParallel0=

;checkBlocksRule1=
;checkBlocksBS1=
;checkBlocksCompr1=
;checkBlocksRead1=
;checkBlocksParallel1=

;checkBlocksRule2=
;checkBlocksBS2=
;checkBlocksCompr2=
;checkBlocksRead2=
;checkBlocksParallel2=

;checkBlocksRule3=
;checkBlocksBS3=
;checkBlocksCompr3=
;checkBlocksRead3=
;checkBlocksParallel3=

;checkBlocksRule4=
;checkBlocksBS4=
;checkBlocksCompr4=
;checkBlocksRead4=
;checkBlocksParallel4=

#  List of Devices for md5 ckeck depending on blocks of these
#  Devices (eg. /dev/sdb or /dev/sdb1)
;checkDevices0=

# Directory where to store the backups of the devices
;checkDevicesDir0=

# Block size of option checkDevices0
# default is 1M (1 megabyte)
;checkDevicesBS0=

# if set, the blocks generated due to checkDevices0 are compressed
# possible values: 'check', 'yes' or 'no', default is 'no'
# check users COMRESSION_CHECK (see option compressSuffix)
;checkDevicesCompr0=

# Read devices specified here in parallel to "normal" ones.
# This only makes sense if they are on a different disk.
# Default value is 'no'
;checkDevicesParallel0=

;checkDevices1=
;checkDevicesDir1=
;checkDevicesBS1=
;checkDevicesCompr1=
;checkDevicesParallel1=

;checkDevices2=
;checkDevicesDir2=
;checkDevicesBS2=
;checkDevicesCompr2=
;checkDevicesParallel2=

;checkDevices3=
;checkDevicesDir3=
;checkDevicesBS3=
;checkDevicesCompr3=
;checkDevicesParallel3=

;checkDevices4=
;checkDevicesDir4=
;checkDevicesBS4=
;checkDevicesCompr4=
;checkDevicesParallel4=

# write temporary dbm files in --tmpdir
# use this if you have not enough RAM, default is no
;saveRAM=

# compress command (with options), default is <bzip2>
;compress=

# uncompress command (with options), default is <bzip2 -d>
;uncompress=

# postfix to add after compression, default is <.bz2>
;postfix=

# do not compress files with the following
# suffix (uppercase included):
# (if you set this to '.*', no files will be compressed)
# Default is \\.zip \\.bz2 \\.gz \\.tgz \\.jpg \\.gif \\.tiff? \\.mpe?g \\.mp[34] \\.mpe?[34] \\.ogg \\.gpg \\.png \\.lzma \\.xz \\.mov
;exceptSuffix=

# like --exceptSuffix, but do not replace defaults, add
;addExceptSuffix=


# Like --exceptSuffix, but mentioned files will be
# compressed. If you chose this option, then files not
# affected be execptSuffix, addExceptSuffix or this Suffixes
# will be rated by the rule function COMPRESS_CHECK wether
# to compress or not
;compressSuffix=

# Files smaller than this size will never be compressed but always
# copied. Default is 1024
minCompressSize=10000000

# alternative to exceptSuffix, comprRule and minCompressSize:
# definition of a rule which files will be compressed
# If this rule is set, exceptSuffix, addExceptSuffix
# and minCompressSize are ignored.
# Default rule _generated_ from the options above is:
# comprRule = '$size > 1024' and not
#   '$file =~ /.zip\\Z|.bz2\\Z|.gz\\Z|.tgz\\Z|.jpg\\Z|.gif\\Z|.tiff\\Z|.tif\\Z|.mpeg\\Z|.mpg\\Z|.mp3\\Z|.ogg\\Z|.gpg\\Z|.png\\Z/i'
# or (eg. if compressSuffix = .doc .pdf):
#   '$size > 1024 and not $file =~ /.zip\\Z|.bz2\\Z|.gz\\Z|.tgz\\Z|.jpg\\Z|.gif\\Z|.tiff\\Z|.tif\\Z|.mpeg\\Z|.mpg\\Z|.mp3\\Z|.ogg\\Z|.gpg\\Z|.png\\Z/i and ( $file =~ /.doc\\Z|.pdf\\Z/i or &::COMPRESSION_CHECK($file) )'
;comprRule='$size > &::Size("100M")'

# maximal number of parallel compress operations,
# default = choosen automatically
;noCompress=

# length of queue to store files before compression,
# default = 1000
;queueCompress=

# maximal number of parallel copy operations,
# default = 1
;noCopy=

# length of queue to store files before copying,
# default = 1000
;queueCopy=

# write statistics about used space in log file
# default is 'no'
;withUserGroupStat=

# write statistics about used space in name file
#                   will be overridden each time
# if no file name is given, nothing will be written
# format is:
# identifier uid userName value
# identifier gid groupName value
;userGroupStatFile=

# default is 'no', if you do not want to compress, say 'yes'
;doNotCompressMD5File=

# permissions of .md5checkSumFile, default is 0600
;chmodMD5File=

# verbose messages, about exceptRule and includeRule
# and added files. default is 'no'
;verbose=

# generate debug messages, levels are 0 (none, default),
# 1 (some), 2 (many) messages
;debug=

# reset access time in the source directory - but this will
# change ctime (time of last modification of file status
# information
# default is 'no', if you want this, say 'yes'
;resetAtime=

# do not delete any old backup (e.g. specified via --keepAll or
# --keepWeekday) but print a message. This is for testing configuratons
# or if you want to delete old backups with storeBackupDel.pl.
# Values are 'yes' and 'no'. Default is 'no' which means to not delete.
doNotDelete=no

# delete old backups which have not been finished
# this will not happen if doNotDelete is set
# Values are 'yes' and 'no'. Default is 'no' which means not to delete.
;deleteNotFinishedDirs=

# keep backups which are not older than the specified amount
# of time. This is like a default value for all days in
# --keepWeekday. Begins deleting at the end of the script
# the time range has to be specified in format 'dhms', e.g.
# 10d4h means 10 days and 4 hours
# default = 30d;
# An archive flag is not possible with this parameter (see below).
;keepAll=9999999999999999999999d

# keep backups for the specified days for the specified
# amount of time. Overwrites the default values choosen in
# --keepAll. 'Mon,Wed:40d Sat:60d10m' means:
# keep backups from Mon and Wed 40days + 5mins
# keep backups from Sat 60days + 10mins
# keep backups from the rest of the days like spcified in
# --keepAll (default 30d)
# you can also set the 'archive flag'.
# 'Mon,Wed:a40d5m Sat:60d10m' means:
# keep backups from Mon and Wed 40days + 5mins + 'archive'
# keep backups from Sat 60days + 10mins
# keep backups from the rest of the days like specified in
# --keepAll (default 30d)
# If you also use the 'archive flag' it means to not
# delete the affected directories via --keepMaxNumber:
# a10d4h means 10 days and 4 hours and 'archive flag'
;keepWeekday=

# do not delete the first backup of a year
# format is timePeriod with possible 'archive flag'
;keepFirstOfYear=

# do not delete the last backup of a year
# format is timePeriod with possible 'archive flag'
;keepLastOfYear=

# do not delete the first backup of a month
# format is timePeriod with possible 'archive flag'
;keepFirstOfMonth=

# do not delete the last backup of a month
# format is timePeriod with possible 'archive flag'
;keepLastOfMonth=

# default: 'Sun'. This value is used for calculating
# --keepFirstOfWeek and --keepLastOfWeek
;firstDayOfWeek=

# do not delete the first backup of a week
# format is timePeriod with possible 'archive flag'
;keepFirstOfWeek=

# do not delete the last backup of a week
# format is timePeriod with possible 'archive flag'
;keepLastOfWeek=

# keep multiple backups of one day up to timePeriod
# format is timePeriod, 'archive flag' is not possible
# default is 7d
;keepDuplicate=

# Keep that miminum of backups. Multiple backups of one
# day are counted as one backup. Default is 10.
keepMinNumber=10000000000

# Try to keep only that maximum of backups. If you have more
# backups, the following sequence of deleting will happen:
# - delete all duplicates of a day, beginning with the old
#   once, except the oldest of every day
# - if this is not enough, delete the rest of the backups
#   beginning with the oldest, but *never* a backup with
#   the 'archive flag' or the last backup
;keepMaxNumber=

# Alternative deletion scheme. If you use this option, all
# other keep options are ignored. Preserves backups depending
# on their *relative* age. Example:
#
#   keepRelative = 1d 7d 61d 92d
#
# will (try to) ensure that there is always
#
# - One backup between 1 day and 7 days old
# - One backup between 5 days and 2 months old
# - One backup between ~2 months and ~3 months old
#
# If there is no backup for a specified timespan (e.g. because the
# last backup was done more than 2 weeks ago) the next older backup
# will be used for this timespan.
;keepRelative =

# print progress report after each 'number' files
# Default is 0, which means no reports.
# additional you may add a time frame after which a message is printed
# if you want to print a report each 1000 files and after
# one minute and 10 seconds, use: -P 1000,1m10s
;progressReport=

# print depth of actual readed directory during backup
# default is 'no', values are 'yes' and 'no'
;printDepth=

# ignore read errors in source directory; not readable
# directories does not cause storeBackup.pl to stop processing
# Values are 'yes' and 'no'. Default is 'no' which means not
# to ignore them
;ignoreReadError=

# after a successful backup, set a symbolic link to
# that backup and delete existing older links with the
# same name
;linkToRecent=

# name of the log file (default is STDOUT)
;logFile=

# if you specify a log file with --logFile you can
# additionally print the output to STDOUT with this flag
# Values are 'yes' and 'no'. Default is 'no'.
;plusLogStdout=

# output in logfile without time: 'yes' or 'no'
# default = no
;suppressTime=

# maximal length of log file, default = 1e6
;maxFilelen=

# number of old log files, default = 5
;noOfOldFiles=

# save log files with date and time instead of deleting the
# old (with [-noOfOldFiles]): 'yes' or 'no', default = 'no'
;saveLogs=

# compress saved log files (e.g. with 'gzip -9')
# default is 'bzip2'
;compressWith=

# write log file (also) in the backup directory:
# 'yes' or 'no', default is 'no'
# Be aware that this log does not contain all error
# messages of the one specified with --logFile!
# Some errors are possible before the backup
# directory is created.
;logInBackupDir=

# compress the log file in the backup directory:
# 'yes' or 'no', default is 'no'
;compressLogInBackupDir=

# filename to use for writing the above log file,
# default is '.storeBackup.log'
;logInBackupDirFileName=



\nwnotused{Versions.conf}\nwendcode{}\nwbegindocs{5}\nwdocspar

\end{comment}